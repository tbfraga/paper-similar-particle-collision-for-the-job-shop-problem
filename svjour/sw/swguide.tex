\NeedsTeXFormat{LaTeX2e}[1995/12/01]

\documentclass[draft]{ltxguide}[1995/11/28]
%\usepackage{draftcopy}

\makeatletter
\def\setitemindent#1{\settowidth{\labelwidth}{#1}%
        \leftmargini\labelwidth
        \advance\leftmargini\labelsep
   \def\@listi{\leftmargin\leftmargini
        \labelwidth\leftmargini\advance\labelwidth by -\labelsep}}
\def\setitemitemindent#1{\settowidth{\labelwidth}{#1}%
        \leftmarginii\labelwidth
        \advance\leftmarginii\labelsep
\def\@listii{\leftmargin\leftmarginii
        \labelwidth\leftmarginii\advance\labelwidth by -\labelsep
        \parsep=\parskip
        \topsep=\z@
        \itemsep=\parskip \advance\itemsep by -\parsep}}
%
% adjusted environment "description"
% if an optional parameter (at the first two levels of lists)
% is present, its width is considered to be the widest mark
% throughout the current list.
\def\description{\@ifnextchar[{\@describe}{\list{}{\labelwidth\z@
          \itemindent-\leftmargin \let\makelabel\descriptionlabel}}}
%
\def\describelabel#1{#1\hfil}
\def\@describe[#1]{\relax\ifnum\@listdepth=0
\setitemindent{#1}\else\ifnum\@listdepth=1
\setitemitemindent{#1}\fi\fi
\list{--}{\let\makelabel\describelabel}}
\DeleteShortVerb{\|}
\renewenvironment{decl}[1][]%
    {\par\small\addvspace{2.3ex}%
     \vskip -\parskip
     \ifx\relax#1\relax
        \def\@decl@date{}%
     \else
        \def\@decl@date{\NEWfeature{#1}}%
     \fi
     \noindent\hspace{-\leftmargini}%
     \begin{tabular}{|l|}\hline\ignorespaces}%
    {\\\hline\end{tabular}\nobreak\@decl@date\par\nobreak
     \vspace{2.3ex plus 1ex}\vskip -\parskip}
\MakeShortVerb{\|}
%
\def\@listI{\leftmargin\leftmargini
            \parskip 0\p@ \@plus 1\p@
            \parsep 0\p@ \@plus\p@
            \topsep 2\p@ \@plus\p@
            \itemsep0\p@\relax}
\@listI
\setlength{\parskip}{\medskipamount}
\makeatother
\newcommand{\SJour}{\textsc{SVJour}}

\title{The \SJour\ document class users guide\\supplement for\\
\textit{ShockWaves}}

\author{\copyright~1998, Springer Verlag Heidelberg\\
   All rights reserved.}

\date{18 May 1998}

\begin{document}

\maketitle

\section{Introduction}
\label{sec:intro}
This document describes the \textit{sw} option for the
\SJour\ \LaTeXe\ document class. For details on
manuscript handling and the review process we refer to the
\emph{Instructions for authors} in the printed journal. For style
matters please consult previous issues of the journal.

\section{Initializing the class}
\label{sec:opt}

As explained in the main \emph{Users guide} you can
begin a document for \emph{Shock Waves} by including
\begin{verbatim}
   \documentclass[sw]{svjour}
\end{verbatim}
as the first line in your text. All other options are also described in
the main \emph{User guide}.

\section{Changes to the \SJour\ class standard}

As the abstract of your article is to appear in the header section,
it must be coded before the \verb|\maketitle| command. Do not use the
\verb|\begin{abstract}| \dots \verb|\end{abstract}| environment of
standard \LaTeX. Instead proceed as you do for the other
front matter declarations:
\begin{decl}
|\abstract| \arg{Text of your abstract}
\end{decl}
The standard key words are also part of the frontmatter please code
them at the end but still inside the \verb|\abstract{...}| area.

\section{Changed bibliographic environment}
The mechanism of explicit labels in \verb|\bibitem| commands has been
changed to reflect those labels only when the actual bibliographic
source is \verb|\cite|d in the text. In the reference list they are
simply suppressed. If you do not use the optional parameter of the
\verb|\bibitem| commands the behaviour is as normal: the items are
numbered consecutively for their citation in the text -- the reference
list itself has no numbers.

This change can be canceled by using the option \verb|oribibl| --
particular useful when you are using \textsc{Bib}\TeX\ that relies on an
unchanged bibliographic/citation apparatus.

There is the additional option \verb|bibself| that supresses the
bracing of \verb|\cite| commands in the text altogether. With it you can
let \LaTeX\ incorporate the content of a \verb|\bibitem|s label
smoothly into the flow of your text (see the example file for
demonstration).

Again we strongly suggest to use the \verb|\bibitem - \cite| as well as
the \verb|\label -| \verb|\ref| mechanism of \LaTeX\ for your cross
references throughout your document.
\end{document}
